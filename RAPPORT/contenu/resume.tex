\chapter*{Résumé}
\phantomsection
\addcontentsline{toc}{chapter}{Résumé}


Ce livre présente la conception et la mise en œuvre d'un système de \textbf{monitoring en temps réel des batteries solaires} dans une installation photovoltaïque. L'objectif est de surveiller et d'améliorer l'efficacité du stockage d'énergie, en fournissant des données précises sur l'état des batteries, telles que la tension, le courant, la température, l'état de charge (SoC) et l'état de décharge (DoD).\\

Le système repose sur des capteurs pour collecter les informations essentielles des batteries. Ces données sont traitées par une unité centrale qui les transmet à une plateforme de gestion à distance. L'analyse de ces informations permet de prendre des décisions en temps réel, comme ajuster la charge ou décharge des batteries pour éviter leur détérioration et prolonger leur durée de vie.\\

Le fonctionnement du système repose sur plusieurs étapes clés :
\begin{itemize}
	\item \textbf{Mesure de la tension et du courant} : ces deux paramètres sont essentiels pour suivre l'état de santé des batteries. Le courant permet de déterminer la quantité d'énergie qui entre ou sort des batteries, tandis que la tension aide à surveiller les niveaux de charge et à détecter les anomalies.
	\item \textbf{Comptage de Coulombs} : cette méthode calcule avec précision l'état de charge (SoC) des batteries en mesurant le courant entrant et sortant sur une période donnée.
	\item \textbf{État de Décharge (DoD)} : ce paramètre évalue le niveau de décharge des batteries, fournissant des indications sur la quantité d'énergie restante et permettant d'optimiser l'utilisation de l'énergie stockée.
	\item \textbf{Analyse des températures} : les variations de température affectent la performance et la sécurité des batteries, et le monitoring des températures permet de prévenir les surchauffes.
	\item \textbf{Gestion de la charge et de la décharge} : en surveillant l'état des batteries, il est possible d'ajuster dynamiquement les cycles de charge et de décharge pour optimiser l'utilisation de l'énergie stockée.
	\item \textbf{Interface de gestion} : une interface utilisateur affiche les données recueillies et permet une surveillance en temps réel des paramètres critiques, déclenchant des alertes en cas de problèmes comme la surcharge, la sous-tension ou la surchauffe.
\end{itemize}

L'objectif final du système est d'assurer une \textbf{gestion efficace et sécurisée} du stockage d'énergie solaire, tout en maximisant la durée de vie des batteries et en minimisant les risques de défaillance.\\

\textbf{Mots-clés} : batteries solaires, monitoring, gestion de l'énergie, stockage, optimisation, SoC, DoD, système photovoltaïque.
