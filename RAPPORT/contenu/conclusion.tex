\cleardoublepage
\markboth{Conclusion générale et perspectives}{Conclusion générale et perspectives} % Ajout de la commande \markboth
\chapter*{Conclusion générale et perspectives}
\addcontentsline{toc}{chapter}{Conclusion générale et perspectives}

Au terme de notre étude sur le système de surveillance des batteries solaires, nous avons mis en lumière l'importance cruciale des systèmes photovoltaïques et des nombreux facteurs influençant leur performance. La qualité des panneaux solaires, des batteries et des autres équipements, ainsi qu'un dimensionnement approprié, sont essentiels pour garantir un fonctionnement optimal. La nécessité d'une surveillance régulière des paramètres tels que la température, la tension et le courant a été soulignée, démontrant l'importance d'un dispositif de monitoring efficace.

Nous avons également exposé le principe de fonctionnement de notre dispositif, en mettant en avant la sélection minutieuse des composants électroniques. Chaque élément a été choisi en fonction de ses caractéristiques techniques et de son adéquation avec les exigences de notre système de surveillance, ce qui est fondamental pour assurer la fiabilité de l'ensemble.

La transmission des données a été détaillée en utilisant le Wi-Fi, les protocoles HTTP/HTTPS ainsi que le protocole MQTT, garantissant une communication sécurisée entre le dispositif et la plateforme, tout en assurant une transmission efficace vers une base de données en ligne. Par ailleurs, nous avons procédé à la modélisation de la base de données en concevant une structure optimisée sous MySQL pour le stockage des données collectées, ce qui facilite l'accès et l'analyse des informations.

Le dimensionnement des composants et les connexions entre le microcontrôleur ESP32 et les autres éléments du système ont également été explorés, illustrant la cohérence du système. Ces connaissances constituent une base solide pour les prochaines étapes, notamment la réalisation du tableau de bord et l'implémentation des tests.

En perspectives, nous envisageons d'approfondir l'analyse des données recueillies pour optimiser encore davantage la gestion des batteries solaires. De plus, l'ajout de fonctionnalités intelligentes, comme la prédiction des performances basées sur l'historique des données, pourrait enrichir notre système. Enfin, des tests sur le terrain permettront de valider l'ensemble des performances et d'ajuster le dispositif selon les besoins spécifiques des utilisateurs.
