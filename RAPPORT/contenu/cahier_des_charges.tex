\addcontentsline{toc}{chapter}{Cahier des charges}
%\chapter*{Cahier des charges}
\begin{titlepage}
	


	\begin{center}
		Projet de mémoire LICENCE  AU 2021 / 2022
	\end{center}

	\begin{center}
		\large{\textbf{REGISTRE DE PRÉSENCE NUMÉRIQUE}} \\
	\end{center}

	\begin{normalsize}
		\noindent\textbf{Contexte :} \\
		\indent Une grande école se doit d'avoir des suivis sérieux sur les présences et les absences des enseignants et des étudiants dans son établissement. Ces suivis sont réalisés grâce aux cahiers de texte et aux fiches de présence. Mais, les cahiers de texte version papier sont difficiles à gérer et se perdent très facilement. De plus, ces derniers sont obsolètes à cause de l’évolution des technologies. La mise en œuvre d’un registre de présence numérique est alors une solution pour remédier à tout cela. \\
	\end{normalsize}

	\begin{normalsize}
		\noindent\textbf{Objectifs :} \\
		\indent Avoir des suivis précis sur le nombre d’absences des étudiants, les heures de présence des enseignants et l’évolution des programmes universitaires. Remplacer les signatures des enseignants par le scan de son leur code-barres.
\\
	\end{normalsize}

	\begin{normalsize}
		\noindent\textbf{Travaux demandés :} 
		
		\begin{itemize}
			\item Documentations sur les outils
\vspace{3mm}
			\item Développement d’un logiciel générateur de code-barre 
\vspace{3mm}
			\item Mise en œuvre d’une application mobile
\vspace{3mm}
			\item Test et validation
\vspace{3mm}
			
		\end{itemize}
	\end{normalsize}

	\begin{normalsize}
		\noindent\textbf{Lieu de travail :} 
		
 		\indent Université d'Antsiranana - Laboratoire électronique \\
	\end{normalsize}

	\begin{normalsize}
		\noindent\textbf{Encadreurs :}
		
		\indent Dr. RAJONIRINA Solofanja Jeannie \\
		\indent Mr. RAMANAN’HAJA Hery Tina
	
	\end{normalsize}

\end{titlepage}