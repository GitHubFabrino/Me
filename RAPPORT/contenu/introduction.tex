\chapter*{Introduction générale}
\phantomsection
\addcontentsline{toc}{chapter}{Introduction générale}  

La surveillance des batteries solaires est devenue une nécessité incontournable dans la gestion moderne des systèmes photovoltaïques. Depuis l'émergence des technologies solaires, le besoin de stocker l'énergie produite pour une utilisation ultérieure a conduit à d’importantes avancées dans la conception et la gestion des batteries. Ces systèmes de stockage jouent un rôle essentiel en assurant un approvisionnement énergétique constant, même en l’absence de production solaire.  


Au fil des années, diverses techniques et outils ont été développés pour améliorer la gestion des batteries solaires, notamment grâce à la surveillance en temps réel de leurs paramètres clés. Ces progrès résultent d’efforts continus pour mieux comprendre et optimiser le fonctionnement des batteries, en intégrant des solutions technologiques avancées permettant de mesurer la tension, le courant, la température ainsi que les états de charge et de décharge.  


Bien que les installations solaires se multiplient, la gestion efficace des batteries demeure un défi majeur. Il est crucial de disposer de systèmes de monitoring capables de fournir des données précises et en temps réel afin de maximiser leur durée de vie et garantir des performances optimales. Cela implique une analyse détaillée des courants de charge et de décharge ainsi qu’une surveillance rigoureuse des tensions pour prévenir les défaillances.  


C’est dans ce contexte que ce livre explore les principes fondamentaux du monitoring des batteries solaires. Il débute par une vue d’ensemble des systèmes photovoltaïques et des batteries, en détaillant leurs principes de fonctionnement et caractéristiques. Ensuite, il traite des composants électroniques dédiés à la surveillance et examine les techniques de transmission des données en temps réel ainsi que la modélisation des bases de données, essentielles pour une gestion efficace des informations collectées.

La réalisation pratique du dispositif de surveillance est ensuite abordée, depuis sa conception jusqu’à sa mise en œuvre. La présentation des interfaces de supervision illustre l’exploitation des données issues du monitoring et la gestion du parc. Enfin, le dernier chapitre se conclut par une analyse du traitement des données et la conception d’une intelligence artificielle destinée à prédire la durée de vie restante ainsi que l’état de santé des batteries.
