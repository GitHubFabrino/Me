
\addcontentsline{toc}{chapter}{Cahier des charges}
	\begin{minipage}{0.15\linewidth}		
		\begin{flushleft}
			\includegraphics[width=3cm,height=2.5cm]{./img/stic.png}
		\end{flushleft}
	\end{minipage}
	\hfill
	\begin{minipage}{0.85\linewidth}		
		\begin{center}
			\begin{tabular}{c}				
				École Supérieure Polytechnique d'Antsiranana \\
				Mention Science et Technologie de l'Information et de la communication \\
				B.P.O 201 - ANTSIRANANA - MADAGASCAR \\
				Tel. : +261 (0) 32 76 395 40 Courriel : mentionsticespa@gmail.com \\
			\end{tabular}
		\end{center}		
	\end{minipage}
	\hfill	
	\rule{17cm}{0.1mm}
	\begin{flushright}
		 «Maîtriser aujourd'hui la technologie de demain.»
	\end{flushright}


	\begin{center}
		\textcolor{black}{\large{\textbf{Monitoring en temps réel des batteries solaires
}}} \\ \vspace{0.4cm}
		\textcolor{black}{\large{\textbf{(Proposition : 3 batteries)}}} \\
		
	\end{center}

	\begin{normalsize}
		\noindent\textbf{Objectif :} \\
		
		\indent Il s’agit de concevoir et réaliser un dispositif permettant de mesurer les paramètres d’un parc de batteries d’une centrale photovoltaïques, de transmettre ces données dans une base de données en ligne et de développer un tableau de bord permettant de surveiller l’évolution de ces paramètres.
\\
		
	\end{normalsize}

	\begin{normalsize}
		\noindent\textbf{Contexte :} \\
		
		\indent Le dispositif peut mesurer les paramètres caractéristiques des batteries (tension de charge et décharge de batteries, capacité…) et afficher en temps réel les données sur l’état des batteries.
		Le tableau de bord peut disposer plusieurs fonctionnalités. 
		Il devrait être capable de :
		\begin{itemize}
			\item Paramétrer l’intégralité des fonctions et des seuils de fonctionnement de la batterie (plage de température, plage de tension, plage de courant, plage de puissance, seuils d’alerte, profondeur de décharge maximale, etc.)
			\item	D’afficher l’intégralité des données de fonctionnement de la batterie forme de graphiques et tableaux
		\item	De montrer les incidents en avance de phase
			\item	Gérer le parc et le maintenance préventive
			\item	Compresser les données sur 5 niveaux en fonction de leur ancienneté (vérification et gestion de l’historique de données)
			\item	Conserver indéfiniment les valeurs moyennes, minimales et maximales de chaque paramètre. 
			\item	Gérer le système de sécurité de données `à l’aide des profils utilisateurs et droits d’accès pour la consultation.
			\item	Crypter toutes les données Stockées pour la confidentialité des utilisateurs.
			\item	D’envoyer une alerte SMS, email ou notification pour être déclenchée en fonction des paramètres définis par l’utilisateur.
\\
		\end{itemize}
		
		\noindent L’étudiant travaillera au sein de Lab’Vision pour bénéficier de ses ressources, notamment l’emprunt de matériel et d’autres avantages, afin de concrétiser leur conception.
\\
		
	\end{normalsize}

	
\newpage
	\begin{normalsize}
		\noindent\textbf{Travaux demandés :} \\
		
		\textbf{Projet de fin de semestre :}
		\begin{itemize}
			\item Étude et analyse de fonctionnement des batteries
			\item	Schéma bloc et fonctionnel du système
			\item	Identifier les composants
			\item	Réalisation et Test du dispositif pour collecter les données des centrales PV
			\item	Modélisation et simulation des batteries en bonne et mauvaise état
\\
			
		\end{itemize}
	
		\textbf{Projet de mémoire :}
\\
		\begin{itemize}
			\item Identifications des différentes fonctionnalités de l’ensemble du système 
			\item	Modélisation de l’application (UML)
			\item	Conception et développement du tableau de bord 
			\item	Test de l’ensemble
\\
			
		\end{itemize}
	\end{normalsize}


	\begin{normalsize}
		\noindent\textbf{Encadreurs :} \\
		\indent Dr. RAJONIRINA Solofanja Jeannie\\
		\indent M. ANDRIANIVOSOA Ramamonjy\\
		\indent Dr. RAMANAN'HAJA Hery Tina\\
		\indent Mme RASOARIMANANA Aicha Yvanna\\
		
	\end{normalsize}
