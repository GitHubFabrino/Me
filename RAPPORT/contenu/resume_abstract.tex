\chapter*{Résumé}
\phantomsection
\addcontentsline{toc}{chapter}{Résumé}


\indent Le registre de présence numérique est mis en œuvre grâce à deux plateformes informatiques. D’abord, un logiciel générateur de carte à code-barres pour les enseignants et les étudiants, et permet de visualiser les listes des enseignants et des étudiants en fonction de leurs parcours. Le code-barres qui représentera les informations de chaque individu sous formes de barres verticales. Ensuite, une application mobile scanneur de codes-barres qui permettra de réaliser la présence des enseignants et des étudiants. 
\\

Les deux technologies sont développées en hybride en utilisant Flutter qui est un SDK du langage de programmation Dart. Flutter est multiplateforme, donc on peut coder avec lui pour la création des applications hybrides (Android, iOS), des logiciels (Windows, Linux, MacOS) ainsi que des sites web. Notre application fonctionnera sur Android et Windows pour le logiciel à cause de leur popularité.
\\

Les données qui seront traitées vont être stockées dans un serveur local XAMPP (pour le test), avec un SGBD MySQL pour la gestion des tables de données.  



\vspace{1cm}

%\chapter*{Abstract}
{\Huge \textbf{Abstract}} \\ 

\indent The digital presence register is implemented thanks to two computer platforms. First, a barcode card generator software for teachers and students and allows you to view lists of teachers and students according to their courses. The barcode that will represent the information of each individual in the form of vertical bars. Next a mobile application barcode scanner, that will make it possible to achieve the presence of teachers and students.
\\

Both technologies are developed in hybrid using Flutter, which is a DART programming language SDK. Flutter is multiplatform so we can coder with it for the creation of hybrid applications (Android, IOS), software (Windows, Linux, macOS) as well as websites. Our application will work on Android and Windows for the software because of their popularity.
The data that will be processed will be stored in an XAMPP local server (for the test), with a MySQL DBMS for managing data tables.