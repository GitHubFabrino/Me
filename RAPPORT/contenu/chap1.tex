\chapter{Etat de l'art sur la surveillance environnementale}
\section{Introduction}
Ce premier chapitre présente trois volets. Le premier expose les enjeux de la surveillance de déforestation, en présentant les impacts écologiques de la déforestation, à savoir sur la gestion des ressources naturelles, la biodiversité et l’impact socio-économique et développement. Le deuxième présente deux technologies utilisées dans la surveillance environnementale liées à la déforestation, telle que la télédétection et l’imagerie satellite et la surveillance par drone. Puis, il présent diverses méthodes actuelles utilisés à Madagascar pour la surveillance des aires protégés. Dans la conclusion de ce chapitre, est présentée succinctement la méthode proposée dans cette thèse ainsi que l’argumentation pour la mettre en évidence par rapport aux autres méthodes précédemment citées.

\section{Enjeux de la surveillance de déforestation }
Au fil des décennies, la déforestation s'est intensifiée, alimentée par une multitude de facteurs socio-économiques, allant de la culture de cultures lucratives à l'exploitation forestière incontrôlée pour répondre à la demande croissante de produits du bois.
\\

\subsection{Cause socio-économique de la déforestation dans la région nord de Madagascar}
\subsubsection{Culture excessive de Catha edulis}
Au nord de Madagascar, selon les conclusions de l'équipe de l'Association MNP, l'une des principales causes de déforestation autour de la réserve naturelle de la Montagne d'Ambre est la culture du khat. Avant le départ des Français dans les années 1960-1970, les communes de Joffre Ville, situées dans la région du parc, étaient parmi les principales productrices de khat, approvisionnant l'ensemble de Madagascar. Les agriculteurs malgaches engagés dans ce commerce plantent la célèbre Catha edulis dans l'espoir d'améliorer leurs conditions de vie. 
\\

Cependant, cette pratique détruit rapidement les zones boisées environnantes du parc, mettant en péril la couverture forestière. Si les bois environnants venaient à être détruits, les prochaines cibles des exploitants forestiers pourraient être les bois de la Montagne d'Ambre. Une intrusion progressive dans la zone protégée est déjà observée. De manière alarmante, ce phénomène prend de l'ampleur, la demande de khat augmente tant à Antsiranana que dans d'autres grandes villes du pays. Les prix grimpent, les producteurs s'enrichissent, et de nombreux agriculteurs, attirés par les profits, se lancent dans la production sans se rendre compte des conséquences néfastes qui en résultent \cite{3}.
\\

\subsubsection{Abus d’exploitation forestière pour la production de charbon de bois et pour l’utilisation de bois de chauffe}
La production de charbon de bois est un moyen de subsistance important pour les communautés rurales d'Afrique et de Madagascar, offrant une source de revenus pendant la contre-saison agricole et fournir un filet de sécurité en cas de chocs tels que de mauvaises récoltes. La pêche est également très variable dans ses revenus, mais elle est moins risquée car les investissements initiaux sont plus faibles, le retour sur investissement est connu rapidement et la demande de produits de la pêche est forte à Antsiranana. Même si la production de charbon de bois comporte également des risques, notamment des risques sanitaires et (pour les producteurs dépourvus de permis) un risque de confiscation, le charbon de bois peut être produit toute l’année et bénéficie d’une demande relativement continue et de prix \cite{4}, \cite{5}.
\\

À Madagascar, la biodiversité dans les forêts sèches, notamment au sein des aires protégées (AP), est gravement menacée. À Bobaomby, les impacts écologiques sont évidents, avec la disparition des arbres dans la savane d'Ambodimadiro et la réduction de la taille de la forêt de Beantely, constatée régulièrement par les témoignages recueillis. Bien que la savane de l'Andohazompona conserve actuellement suffisamment d'arbres pour la production, il existe un risque imminent de surexploitation des arbres de savane et d'utilisation ultérieure de la forêt, surtout si les autres moyens de subsistance restent limités. Il est probable que, en l'absence de changements dans les moyens de subsistance, la production de charbon de bois reviendra probablement à Baie de courrier. Cette pratique non durable menace à la fois les forêts de l'AP de Bobaomby et les revenus futurs des populations dépendantes de celles-ci. Cependant, la transition vers des sources alternatives de bois prendra plusieurs années, rendant une exclusion stricte de l'utilisation des forêts pour la production de charbon de bois peu réalisable ou appropriée initialement en raison des coûts que cela imposerait aux communautés locales \cite{4}.
\\

\subsubsection{Abus d’exploitation forestière pour la fabrication de meubles, d’immobilier et exportation}
En outre, il est important de souligner l'exploitation abusive des arbres pour la fabrication de meubles et le secteur immobilier, notamment en vue de l'exportation. Bien que ces pratiques répondent souvent à la demande de meubles de haute qualité, elles posent de sérieux risques de déforestation, suscitant des préoccupations majeures concernant la durabilité environnementale. Les bois requis pour la confection de meubles de grande qualité, comme le palissandre, sont généralement caractérisés par une croissance lente et une maturité tardive, nécessitant des décennies pour atteindre la qualité souhaitée \cite{6}.
\\

Cette situation crée une disparité entre la rapidité d'abattage de ces arbres et la lenteur de leur régénération naturelle. Par conséquent, des espèces précieuses telles que le palissandre sont actuellement menacées en raison de cette exploitation intensive et illicite. Selon une enquête de Traffic relayée par RFI, au cours des cinq dernières années, au moins 350 000 arbres ont été abattus illégalement dans des aires protégées, et au moins 150 000 tonnes de rondins ont été exportées illégalement, à 98 \% vers la Chine \cite{7} . Le déséquilibre entre la demande croissante de meubles haut de gamme et la capacité limitée de la nature à régénérer ces ressources souligne l'urgence de repenser nos pratiques d'exploitation forestière et de promouvoir des alternatives durables pour préserver notre précieuse biodiversité et assurer la pérennité des écosystèmes forestiers.
\\

\subsection{Impact écologique de la déforestation}
\subsubsection{Impact sur le changement climatique}
La forêt présente un aspect essentiel dans la régulation du climat, et la déforestation va à l'encontre de cette fonction, mettant en péril sa capacité régulatrice \cite{8}. Les forêts possèdent aussi une capacité essentielle dans l'absorption du dioxyde de carbone (CO2) de l'atmosphère, contribuant ainsi à atténuer les effets du changement climatique. En effet, les arbres absorbent le CO2 lors de la photosynthèse, agissant comme des puits de carbone naturels. Cependant, la déforestation libère ces réserves de carbone stockées, augmentant ainsi les niveaux de gaz à effet de serre dans l'atmosphère \cite{9}. 
\\

Entre août 2006 et juillet 2016, la dégradation de la forêt amazonienne brésilienne a affectée une superficie de 1 869 800 hectares. Le résultat de l’analyse a montré que 13 \% de la zone dégradée a fini par être défrichée et convertie au cours de la période. Parmi cette quantité, 61 \% de la zone dégradée totale n'a connu qu'un seul événement de dégradation sur toute la période. Les émissions nettes se sont élevées à 5,4 GtCO2. Cette valeur est obtenue en tenant compte des émissions dues à la dégradation des forêts et à la déforestation ainsi qu’à l'absorption due à la régénération des forêts dégradées et à la dynamique de la végétation secondaire. D’un côté, cette analyse indique également que la régénération des forêts dégradées a absorbé 1,8 GtCO2 pendant la même période \cite{10}.
\\

En conséquence, la régulation climatique fournie par les forêts est compromise, accentuant les défis liés aux fluctuations climatiques mondiales. Cette relation délicate entre la forêt en tant que régulateur climatique et la déforestation souligne l'urgence de mettre en place des mécanismes de surveillance avancés pour contrôler et prévenir la perte continue de nos précieuses zones boisées.
\\

\subsubsection{Impact sur la qualité des sols et l’agriculture}
La protection du sol par la forêt est un service éco systémique essentiel souvent négligé. Les arbres participent dans la préservation de la stabilité du sol en agissant comme une barrière naturelle contre l'érosion. Leurs racines contribuent à maintenir la structure du sol, prévenant ainsi la dégradation due aux intempéries et aux précipitations. En revanche, la déforestation expose le sol à une vulnérabilité accrue, car elle élimine cette protection naturelle. Cette affirmation a été prouvée par une étude d’impact de la déforestation sur l'érosion des sols dans les régions montagneuses et les zones agricoles de l'ouest de l'Éthiopie.
\\

Dû à la déforestation, le tableau suivant présente la perte annuelle moyenne de sol dans le bassin versant du Haut Anger Ethiopie ainsi que la perte annuelle en terre agricole de sol durant les années 1989, 2002 et 2020 \cite{11}. 
\\

\begin{table}[H]
	\centering
	\caption{Perte de sol dans le bassin versant du Haut Anger Ethiopie}
	\vspace{5mm}
	\begin{tabular}[c]{|>{\centering\arraybackslash}p{2.5cm}|>{\centering\arraybackslash}p{7cm}|>{\centering\arraybackslash}p{5.5cm}|}
		\hline
		\rule[0.5cm]{0cm}{0cm} Année & Perte moyenne de sol dans le bassin versant du Haut Anger (Tonne/hectare/an) & Perte annuelle en terre agricole (Tonne/hectare/an) \\
		\hline
		\rule[0.5cm]{0cm}{0cm} 1989 & 44 & 75,9 \\
		\hline
		\rule[0.5cm]{0cm}{0cm} 2002 & 66,4 & 98,5 \\
		\hline
		\rule[0.5cm]{0cm}{0cm} 2020 & 87,9 & 103,8 \\
		\hline
	\end{tabular}
\end{table}

Par rapport à la superficie impactée par cette dégradation est illustré dans le tableau suivant :

\begin{table}[H]
	\centering
	\caption{Superficie impactée par la perte du sol}
	\vspace{5mm}
	\begin{tabular}[c]{|>{\centering\arraybackslash}p{2.5cm}|>{\centering\arraybackslash}p{6cm}|>{\centering\arraybackslash}p{5.5cm}|}
		\hline
		\rule[0.5cm]{0cm}{0cm} Année & Superficie d’érosion (Kilomètre carré) & Pourcentage de la superficie dégradé \\
		\hline
		\rule[0.5cm]{0cm}{0cm} 1989 & 551,8 & 29,5 \% \\
		\hline
		\rule[0.5cm]{0cm}{0cm} 2002 & 821,6 & 44 \% \\
		\hline
		\rule[0.5cm]{0cm}{0cm} 2020 & 1\,043,8 & 55,8 \% \\
		\hline
	\end{tabular}
\end{table}




Non seulement, l’espace cultivable est réduit, mais la qualité du sol est affectée comme ce qui a été révélé par l’étude des sols dans les forêts communautaires du Petit Himalaya d'Abbottabad, au Pakistan. Dans cette étude, des échantillons de sol ont été collecté venant de trois zones forestières : dense, modérée moindre. Ils ont été traités pour évaluer la densité apparente (BD), la teneur en humidité du sol (SMC), la capacité de rétention d'eau (WHC), le pH, la conductivité électrique (CE), le carbone organique du sol (COS) et les nutriments solubles.  Il a été constaté que la CE, le COS et les éléments nutritifs étaient non seulement plus élevés dans les sols des zones à végétation dense, mais également dans les sols superficiels de toutes les zones de végétation \cite{12}.
\\

Tout ceci conclu que la perte du couvert forestier peut entraîner une érosion accélérée et l'augmentation des risques de glissements de terrain.  De plus, la dégradation résultant de la déforestation peut diminuer la fertilité du sol et a des répercussions graves sur l'agriculture. Ainsi, la vigilance nécessaire à travers la surveillance forestière émerge comme le rempart essentiel pour garantir la santé des sols et préserver la base fondamentale de notre existence.
\\

\subsubsection{Impact sur la conservation de l’eau}
La conservation de l'eau par la forêt est une fonction vitale qui influence directement les cycles hydrologiques régionaux. Les arbres, par le biais de la transpiration, libèrent de la vapeur d'eau dans l'atmosphère. Ce processus est connu sous le nom d’évapotranspiration \cite{13} , \cite{14}
. Cette libération d'eau contribue à la formation de nuages et de précipitations. Elle favorise ainsi la régulation des précipitations et le maintien de l'humidité atmosphérique \cite{15}. De plus, les racines des arbres agissent comme des filtres naturels, retenant l'eau dans le sol et rechargeant les nappes phréatiques \cite{16}.
\\


\begin{figure}[H]
	\centering
	\includegraphics[width=10cm]{./img/2.png}
	\caption{Illustration du cycle de l'eau
	}
\end{figure}

Cependant, la déforestation perturbe ce processus de conservation de l'eau. En retirant les arbres, la capacité de la forêt à réguler le cycle de l'eau est compromise. La diminution de l'évapotranspiration réduit la quantité d'eau libérée dans l'atmosphère, ce qui peut entraîner des modifications dans les schémas des précipitations et une augmentation du ruissellement. 
\\

En 2019, lors d’une étude empirique concernant la déforestation au Malawi, il a été constaté qu’une augmentation de 1,0 point de pourcentage de la déforestation diminue l’accès à l’eau potable de 0,93 point de pourcentage. De cet effet, la déforestation au cours de la dernière décennie dans le pays qui a atteint 14\% a eu un effet sur l’accès à l’eau potable ainsi qu’une diminution de 9\% des précipitations \cite{17}.
\\

En suivant les changements de la couverture forestière, la surveillance fournit des données essentielles pour évaluer l'ampleur des activités de coupe et anticiper les conséquences sur les cycles hydrologiques régionaux. Grâce à une surveillance continue, il devient possible de mettre en place des mesures de conservation spécifiques, de promouvoir la reforestation, et ainsi de sauvegarder la stabilité des écosystèmes hydriques et garantir la disponibilité d'eau à long terme \cite{18}.
\\

\subsubsection{Impact de la déforestation sur la biodiversité}
Les forêts abritent une diversité incroyable d’espèces végétales et animales. La découpe d'arbre peut entraîner la destruction d'une diversité d'espèces végétales et animales. Tel représenté par le tableau suivant, sont la répartition des espèces qui se trouve à Madagascar d’après le recensement de 2021 par l’IUCN :
\begin{table}[H]
	\centering
	\caption{Répartition des espèces à Madagascar en 2021
}
	\vspace{5mm}
	\begin{tabular}[c]{|>{\centering\arraybackslash}p{5.5cm}|>{\centering\arraybackslash}p{5cm}|>{\centering\arraybackslash}p{5cm}|}
		\hline
		\rule[0.5cm]{0cm}{0cm} Catégorie & Espèces animales & Espèces végétales \\
		\hline
		\rule[0.5cm]{0cm}{0cm} Nombre d'espèce & 352 & 636 \\
		\hline
		\rule[0.5cm]{0cm}{0cm} Habitation en milieu terrestre & 283 & 618 \\
		\hline
		\rule[0.5cm]{0cm}{0cm} Habitation en milieu forestier & 252 & 516 \\
		\hline
		\rule[0.5cm]{0cm}{0cm} Taux Milieu Forestier / Milieu Terrestre & 89\% & 83\% \\
		\hline
	\end{tabular}
\end{table}

On constate alors une forte concentration d’habitation des animaux dans le milieu forestier. Par rapport à cela, la déforestation peut entraîner une perte irréversible de la richesse biologique dans la forêt. Les processus naturels de régénération sont souvent compromis en raison de la fragmentation et de la destruction des habitats des animaux \cite{19}. 
\\

Des espèces endémiques adaptées à des conditions spécifiques de la forêt, peuvent disparaître définitivement. Des arbres majestueux aux plantes herbacées, des oiseaux chanteurs aux mammifères forestiers, la diversité biologique de la forêt est complexe et interconnectée. Notre consultation de la liste noire des espèces menacées de l’IUCN, nous a donné la statistique dans la figure suivante. 78 sur 252 espèces animales et 319 sur 516 espèces végétales habitant dans la forêt sont menacés par l’intrusion et/ou par la perturbation humaine \cite{20}. 
\\

Les espèces menacées sont celles répertoriées comme étant :
\begin{itemize}
	\item En danger critique d'extinction (CR - Critically Endangered), 
	\item	En danger (EN - Endangered), 
	\item	Vulnérables (VU - Vulnerable), 
	\item	Quasi-menacée (NT-Nearly Threatened), 
	\item	ou Moins concerné (LR/LC - Lower Risk or Least Concerned).
\end{itemize}


\begin{figure}[H]
	\centering
	\includegraphics[width=15cm]{./img/3.png}
	\caption{Statistique de la liste noire des espèces (a) végétales et (b) animales menacées de l'IUCN}
\end{figure}

La disparition de certaines espèces peut déclencher des réactions en chaîne, affectant d'autres organismes et perturbant les équilibres écologiques. La biodiversité, une fois perdue, ne peut pas toujours être restaurée dans son état d'origine \cite{19}.
\\

La déforestation peut induire des changements significatifs dans le comportement des espèces animales. La perturbation de l'habitat peut influencer les schémas migratoires, les modèles d'alimentation et même les comportements reproducteurs \cite{21} , \cite{22}. Certains animaux peuvent réagir à la perte d'habitat en modifiant leurs schémas d'accouplement, leurs sites de nidification ou leurs territoires de chasse. Ces changements peuvent avoir des conséquences directes sur la dynamique des populations et la diversité génétique des espèces, compromettant ainsi leur capacité à s'adapter aux pressions environnementales changeantes. Ainsi, la surveillance environnementale devient un outil essentiel pour garantir la préservation des espèces végétales et animales, protégeant ainsi la richesse biologique des forêts pour les générations futures.
\\

\subsection{Impact socio-économique et protection de l’environnement}
\subsubsection{Revenue économique par le tourisme}
Le tourisme a un impact significatif sur divers secteurs économiques. Les secteurs du transport, de l'hébergement, de la restauration et des activités récréatives prospèrent dans les régions qui attirent les touristes en quête d'expériences naturelles \cite{23}. Les communautés locales bénéficient également de cette activité économique, créant des opportunités d'emploi dans le domaine du tourisme et soutenant des initiatives de conservation et de développement durable \cite{24}. Durant l’année 2019, la part de secteur tourisme dans le produit intérieur brut de Madagascar est de 10\%. Le secteur du tourisme a pu générer plus de 520 millions de dollar dans la recette fiscale. Sur le marché de l’emploi, l’industrie touristique génère 750 000 emplois, ce qui représente 12\% de marché de l’emploi de la grande Île \cite{25}.
Ci-joint la statistique de visite des aires protégés gérés par le Madagascar National Parks par rapport aux nombres des touristes à Madagascar.

\begin{figure}[H]
	\centering
	\includegraphics[width=15cm]{./img/4.png}
	\caption{Statistique de visite des aires protégés gérés par le Madagascar National Parks}
\end{figure}

En analysant cette image, on voit bien l’attirance des touristes à visiter les parcs. Or la plupart des parcs de Madagascar sont des zones forestières et naturelles \cite{26}.
\\

Ainsi, la défaillance de la gestion de ces zones protégées peut altérer l’attrait des parcs, entrainant une baisse des revenus touristiques et des répercussions économiques sur les populations locales. Les fonds provenant des visiteurs ne circulent pas aussi librement. Ainsi, s’investir dans la surveillance forestière contribue à préserver le secteur touristique, l’apport financier pour la population locale.

\subsubsection{Avantage économique sur la diversité biologique}

La Convention sur la Diversité Biologique (CDB) établie en 1992 lors du Sommet de la Terre à Rio de Janeiro, est un instrument international majeur qui vise à promouvoir la conservation de la biodiversité à l'échelle mondiale. La CDB encourage spécifiquement des mesures pour assurer la conservation des écosystèmes forestiers. En mettant l'accent sur la nécessité de préserver la variété des formes de vie sur terre, y compris dans les forêts, la CDB reconnaît que la biodiversité forestière contribue non seulement à la santé des écosystèmes locaux mais également au bien-être de la planète dans son ensemble \cite{27},\cite{28}.
\\

L'engagement soutenu de Madagascar envers la CDB, ainsi que son adhésion à long terme aux conventions traitant de la nature en général, telles que la Convention africaine de 1968, Ramsar 1971, CITES 1973, Bonn 1979 et Nairobi 1985, ont incontestablement favorisé les investissements internationaux dans la préservation de la biodiversité dans le pays. Le respect de ces accords présente non seulement une valeur économique, mais aussi des avantages en termes de réputation au sein de la communauté mondiale. En considérant des investissements d'un milliard de dollars sur la période de 1990 à 2020, la conservation de la biodiversité à Madagascar a engendré des avantages moyens annuels de 33,3 millions de dollars (environ 0,23 \% du PIB en 2019). Ceci est équivalant à environ 35,7 millions de dollars annuels en 2021 \cite{24}.
\\

\subsubsection{Accord de Paris sur le climat}
L'Accord de Paris sur le climat, établi en 2015 lors de la COP21, représente une mesure dans la lutte mondiale contre les changements climatiques. Il engage les pays signataires à prendre des actions significatives pour limiter le réchauffement climatique en dessous de 2 degrés Celsius par rapport aux niveaux préindustriels. Dans le contexte des forêts, l'Accord de Paris reconnaît l'importance de la réduction des émissions résultant de la déforestation et de la dégradation des forêts (REDD+). Les pays participants se sont engagés à promouvoir la gestion durable des forêts, à réduire la déforestation et à renforcer les capacités de séquestration du carbone des écosystèmes forestiers \cite{29}. 
\\

Le gouvernement de Madagascar a également souscrit à l'Accord de Paris et présenté son nouveau plan d'action climatique à la Convention-Cadre des Nations Unies sur les Changements Climatiques (CCNUCC). Dans le contexte de la déforestation, la contribution nationale de la République de Madagascar pour atténuer le changement climatique vise une diminution d'environ 30 millions de tonnes de CO2 de ses émissions de gaz à effet de serre d'ici 2030, soit 14\% des émissions nationales par rapport au scénario BAU. Ces projections se basent sur l'inventaire des GES de 2000 à 2010. Cette réduction s'ajoute à l'augmentation prévue de 2 millions de tonnes métriques de CO2 provenant du secteur de l'utilisation des terres, du changement d'affectation des terres et de la foresterie (UTCATF), estimée à 61 millions de tonnes métriques de CO2 d'ici 2030 \cite{30}.

\section{Télédétection par imagerie satellite pour la surveillance forestière} 
\subsection{Principe fondamental de la télédétection par imagerie satellite}
\subsubsection{Les composants de la télédétection par imagerie satellite}
Comme illustré dans la figure \ref{i5}, les éléments essentiels composant l'infrastructure de base de la télédétection sont \cite{31} :
\\

\begin{enumerate}
	\item 	Source d'énergie/illumination : Une source d'énergie est nécessaire pour éclairer l'objet d'intérêt ou lui fournir un rayonnement électromagnétique.
	\item 	Rayonnement/énergie et atmosphère : Tout au long de son trajet, du point de départ à la destination, le rayonnement interagit avec les particules atmosphériques. Une fois que l'énergie parvient de l'objet au capteur, une deuxième interaction peut survenir.
	\item 	Objet d'étude : Lorsque l'énergie atteint enfin sa cible, leur interaction est réglementée par les caractéristiques propres au rayonnement et à l'objet.
	\item 	Capteur enregistreur de rayonnement : Un capteur positionné à distance de l'objet d'étude doit capturer le rayonnement électromagnétique émis ou réfléchi par la cible.
	\item 	Système de traitement des données : Les données enregistrées par le capteur doivent être acheminées (généralement de manière électronique) vers un centre de réception et de traitement, où l'énergie mesurée est convertie en une image utilisable.
	\item 	Analyse et interprétation : Après le traitement des données en télédétection, l'image est soumise à une analyse et une interprétation visuelle et/ou numériques dans le but d'obtenir des informations sur l'objet étudié.
	\item 	Application pratique : En fin de compte, le processus vise à exploiter les informations tirées des images pour approfondir la compréhension de la cible, découvrir des éléments jusqu'alors méconnus, ou contribuer à la résolution de problèmes.
	
\end{enumerate}

\begin{figure}[H]
	\centering
	\includegraphics[width=15cm]{./img/5.png}
	\caption{Processus de télédétection}
	\label{i5}
\end{figure}
\subsubsection{Les capteurs et les données traitées en imagerie satellite}
Dans la littérature, nous avons recensé six types de capteurs utilisés en général \cite{32} ,\cite{33} ,\cite{34} ,  comme illustré dans le tableau suivant :

\begin{table}[H]
	\centering
	\caption{Types de capteurs et leurs utilisations}
	\vspace{5mm}
	\begin{tabular}[c]{|>{\centering\arraybackslash}p{2.5cm}|>{\centering\arraybackslash}p{3.5cm}|>{\centering\arraybackslash}p{4cm}|>{\centering\arraybackslash}p{5cm}|}
		\hline
		\rule[0.5cm]{0cm}{0cm} Types de capteurs & Données traitées & Utilisation & Exemple de capteurs (Début d’activité) \\
		\hline
		\rule[0.5cm]{0cm}{0cm} Optique & Image multispectrales & Cartographie des couleurs, détection de végétation & Landsat 8 (2023), Sentinel-2 (2015), MODIS (Terra-1999, Aqua-2002) \\
		\hline
		\rule[0.5cm]{0cm}{0cm} Radar & Image radar & Cartographie topographique, détection de changement & RADARSAT-2 (2007), Sentinel-1A (2014), Sentinel-1B (2016) \\
		\hline
		\rule[0.5cm]{0cm}{0cm} Hyperspectraux & Spectres complets & Analyse fine des caractéristiques spectrales des surfaces & Hyperion (2000), EnMAP (2021), AVIRIS (1987) \\
		\hline
		\rule[0.5cm]{0cm}{0cm} Thermiques & Rayonnement thermique émis par la surface & Surveillance de la température & Landsat Thermal Infrared Sensors (TIRS, Landsat 8 2023), MODIS Thermal (Terra 1999, Aqua 2002) \\
		\hline
		\rule[0.5cm]{0cm}{0cm} Lidar & Mesures précises de distance & Création de modèles 3D, cartographie topographique & ICESat-2 (2018), CALIPSO (2006) \\
		\hline
		\rule[0.5cm]{0cm}{0cm} De champs magnétiques & Mesures de champs magnétiques & Recherche géophysique & Swarm (2013) \\
		\hline
	\end{tabular}
\end{table}


Le capteur magnétique ne trouve pas d'application dans le domaine de la surveillance environnementale. En revanche, les capteurs optiques, radar, hyperspectraux, thermiques et lidar peuvent être utilisés dans ce contexte, notamment pour la surveillance forestière.
\\

\subsection{Application spécifique à la surveillance forestière}
La télédétection offre plusieurs applications importantes dans le contexte de la déforestation, de la reforestation, des découpes d'arbres illicites et de la biodiversité. 

\subsubsection{Surveillance d’évolution de paysage forestier}
La télédétection par imagerie satellite constitue un outil essentiel pour surveiller de manière dynamique les changements dans le couvert forestier au fil du temps. Les images satellitaires offrent une vue d'ensemble permettant d'identifier les secteurs où la déforestation est en cours. Elle permet ainsi de surveiller l'évolution de ces zones au fil des saisons. En 2020, une méthode de détection de la déforestation appelée Détection Multi temporelle de la Déforestation (MDD) a été proposée. L'idée de base de cette méthode est d'utiliser la différence des valeurs de réflectance sur l'image cible et l'image originale. Deux bandes de Sentinel-2 ont été sélectionnées lors du développement de l'algorithme. Les résultats de la détection de la déforestation obtenus ont une précision très élevée. Par contre, la méthode n’est pas adaptable à tout type de forêts. La résolution étant de 10 mètres à 60 mètres. Le calcul permettra ainsi de calculer la surface déboisée relative aux différences nombre de pixel changeant entre les deux images \cite{35}.
\\

Un aspect particulièrement important de la télédétection est sa capacité à repérer les activités de déforestation illégale. Les anomalies dans le couvert forestier, souvent indicatives de pratiques non autorisées, peuvent être rapidement identifiées, permettant aux autorités de prendre des mesures immédiates pour mettre fin à ces activités préjudiciables. 
\\

En ce qui concerne les initiatives de reforestation, la télédétection est un outil de suivi précis de la croissance des nouvelles plantations d'arbres. Cette surveillance à distance offre une évaluation objective de l'efficacité des programmes de reboisement, permettant aux parties prenantes de mesurer les progrès des projets de restauration forestière. Un outil basé sur Landsat a été utilisé pour constater, l’évolution du couvert forestier de Porto Rico. Le résultat en utilisant des données entre 2000 et 2010 et par rapport à celle de 1991 et 2000 \cite{36}. Le taux de déforestation a diminué de 42,1 \% entre 2000 et 2010, et le reboisement était principalement localisé à l'est et au sud-est de l'île. 

\subsubsection{Surveillance de feux de forêt}
L’imagerie satellite est également largement utilisée pour la détection et la surveillance des feux de forêt. Une étude statistique annuelle permet d’évaluer l’ampleur des feux de forêt et ainsi d’élaborer de stratégie de lutte. Telle est proposée dans une étude à Madagascar en 2022, dont on trouve une quantification des modèles spatio-temporels d’occurrence des incendies à Madagascar à l’aide des données de détection d’incendies par le satellite VIIRS. Les régions de Madagascar où il y a du feu la plus répandue ont été distinguées. Puis, on a su comment le feu évolue au fil du temps et ce que cela a affecté pour les écosystèmes forestiers restants. Une moyenne de 356 189 incendies a été recensée par le système chaque année entre 2012 et 2019, soit une moyenne de 0,604 incendie/km2. Ainsi, cette cartographie par télédétection permet de préciser les zones touchées par les feux de forêt \cite{37},\cite{38}.
Ces données peuvent être combinées à d’autres méthodes qui visent à utiliser des modèles d'apprentissage automatique pour prédire les incendies de forêt un mois à l'avance. Sur la côte Est de Madagascar, l’utilisation d’images satellite de Landsat 7 via Google Earth Engine et les données FIRMS ont montré qu’un modèle de réseau neuronal a atteint une précision de détection d'incendie de 83 \% \cite{39}.
\\

D’autres chercheurs se sont plus axés sur la reconnaissance de la fumée dans les images Landsat-8. Les résultats expérimentaux montrent que le taux de précision de fumée est important. Ceci pourrait segmenter efficacement les pixels dans les images de télédétection. Cette méthode proposée sous les bandes RVB, SWIR2 et AOD peut aider à segmenter la fumée en utilisant une bande de haute sensibilité et un indice de télédétection et déclenche une alarme précoce en cas de fumée de feu de forêt \cite{40}. 
Après l'extinction des incendies, il est possible d'évaluer les impacts à long terme sur l'écosystème en examinant la régénération naturelle de la végétation et la récupération de la biodiversité. En exploitant des données historiques et des images satellite, on peut analyser les facteurs déclenchants des incendies de forêt, tels que les conditions météorologiques, la sécheresse et les activités humaines, afin de mieux comprendre et prévenir ces événements. La télédétection a permis l'utilisation de données open source de Landsat et de pratiques de surveillance des forêts (NDFI ; Z-Scores) pour quantifier les tendances de rétablissement de la forêt sur trois périodes successives de 5 ans. Une analyse des données sur l'occurrence des incendies a été réalisée, et des modèles basés sur les variables bioclimatiques ont été élaborés pour tester les relations avec les tendances de la trajectoire de rétablissement \cite{41}.
\\

\subsubsection{Surveillance des animaux}
L'utilisation de l'imagerie satellite se révèle essentielle pour estimer la taille des populations d'animaux forestiers en suivant leurs schémas de déplacement et en identifiant les zones de plus forte concentration. Cette technologie fournit également des données détaillées sur les interactions entre différentes espèces animales, contribuant ainsi à une compréhension approfondie de l'écologie forestière. Les capteurs radar et optiques à haute résolution de l'imagerie satellite permettent en outre de suivre les migrations d'espèces animales sur de vastes distances, offrant des informations primordiales pour les efforts de conservation \cite{42}.
\\

Avec la télédétection, on trouve des applications spécifiques dans la surveillance des animaux, nécessitant des technologies adaptées pour des fonctions telles que le suivi des migrations. En exemple, des images satellites sont utilisées pour observer la densité géospatiale des lémuriens Lepilemur hubbardorum dans le parc national de Zombitse-Vohibasia. Par ailleurs, la télédétection contribue à la cartographie et à la surveillance des habitats naturels des animaux. Cette utilisation englobe la détection des modifications dans la couverture végétale, la topographie, et d'autres caractéristiques environnementales essentielles pour les espèces concernées. En surveillant les densités et les schémas de comportement des animaux, elle facilite également des études écologiques approfondies, permettant de comprendre les interactions entre différentes espèces animales, ainsi que les facteurs environnementaux qui influent sur leur comportement \cite{43}
\\

\subsection{Analyse des avantages et les limites de l’imagerie satellite sur la surveillance de foret }
\subsubsection{Avantages de la surveillance par satellite}
L'imagerie satellite excelle dans la surveillance forestière en raison de sa portée globale, offrant une couverture complète et régulière des vastes étendues forestières. Cette portée étendue permet d'obtenir une perspective holistique des changements à grande échelle. C’est le cas de l’étude effectué au Soudan, en 2020, utilisant deux types de donnée cloud TM 2000 et Sentinel-2 en 2018. La capacité à collecter des données sur l'ensemble d'un écosystème forestier contribue à une compréhension approfondie des dynamiques environnementales. Cette compréhension fournit ainsi des informations essentielles pour des initiatives de conservation efficaces \cite{44}. 
\\

D'un point de vue temporel, l'imagerie satellite offre une continuité dans la collecte de données, permettant aux chercheurs d'analyser les tendances sur une période prolongée. Par exemple, les données satellites landsat-8 enregistre des données continuellement. Ce satellite est développé pour une durée de vie de 5 ans. Mais en plus, il est lancé avec suffisamment de carburant à bord pour assurer plus de 10 ans d'exploitation. Cette fonctionnalité est particulièrement précieuse pour observer les changements dans les schémas d’événement au fil du temps et de série chronologique bien définie \cite{40}.
\\
L'imagerie satellite présente des avantages considérables en termes de sécurité des données. L'une des principales forces réside dans la nature distante de la collecte de données par satellite, minimisant ainsi les risques liés à la sécurité et à l'intégrité des informations. Les satellites permettent de collecter des données à partir d'altitudes élevées, échappant aux obstacles physiques et aux accès restreints qui pourraient compromettre la sécurité des équipes au sol. Cette approche réduit les risques liés aux activités illégales où des individus pouvant s'opposer à la surveillance. De plus, elle offre une collecte d'informations discrète sans mettre en danger le personnel sur le terrain.
Cet avantage est bien vu pour la surveillance des feux de forêt, dans la capacité à obtenir une vision d'ensemble des incendies à partir d'une position éloignée, éliminant ainsi le besoin d'une proximité physique avec les fronts de feu potentiellement dangereux. Certaines régions forestières peuvent être extrêmement difficiles d'accès en raison de leur topographie complexe, de la densité de la végétation, ou même en raison de conflits locaux ou de situations environnementales dangereuses. Enfin pour la surveillance des animaux, l'imagerie satellite minimise les perturbations, offrant une approche respectueuse de l'environnement pour étudier les espèces dans leur habitat naturel.

\subsubsection{Limites rencontrées en imagerie satellite}
Les limites inhérentes à l'imagerie satellite sont variées et influent sur la capacité à obtenir des données précises et détaillées. L'une des limites majeures de l'imagerie satellite réside dans sa résolution limitée. Les images satellites offrent une vue globale, mais elles peuvent manquer de la finesse nécessaire pour détecter des détails spécifiques, tels que des activités de déforestation à petite échelle ou des changements subtils dans le paysage forestier. Elle ne permet pas d’avoir une précision accrue pour des applications nécessitant une observation minutieuse.
\\

Un autre défi auquel l'imagerie satellite est confrontée concerne les conditions atmosphériques. Les nuages, par exemple, peuvent obstruer la vue depuis l'espace, entraînant des lacunes dans la surveillance. La disponibilité d'images satellites peut être limitée en raison de facteurs météorologiques. De plus, certains satellites peuvent être limités par la période de la journée pendant laquelle les images peuvent être acquises en raison des conditions d'éclairage. Par ailleurs, les satellites ont des cycles d'acquisition d'images qui dépende des orbites et des systèmes de capteurs. Cela peut entraîner des lacunes temporelles entre les captures d'images, rendant difficile la surveillance en temps réel \cite{45}.
\\

Une autre limitation significative de l'imagerie satellite concerne l'accès aux données à traiter. Les données satellitaires, en raison de leur volumétrie et de leur résolution, peuvent nécessiter des capacités de stockage et de traitement massives. Les chercheurs doivent souvent faire face à des défis logistiques pour obtenir et traiter ces données, en particulier dans des régions éloignées ou dans des pays avec une infrastructure informatique limitée.
\\

\subsubsection{Combinaison synergique avec d’autre technologie}
Face aux défis et aux limites de l'imagerie satellite seule, les chercheurs explorent activement des synergies avec d'autres technologies terrestres pour renforcer la surveillance forestière. La combinaison judicieuse de l'imagerie satellite avec des technologies complémentaires offre des avantages significatifs. De plus, la collaboration avec une méthode terrestre vient renforcer cette approche combinée. Les capteurs terrestres ou des patrouilles peuvent fournir des données en temps réel à des échelles locales, complétant les informations globales provenant de l'imagerie satellite et de valider les observations.
\\

D’autres cas d’approche est l'intégration des données prédictives de l'intelligence artificielle avec l'imagerie satellite. Elle réside dans la nécessité de garantir la précision et la fiabilité des modèles d'intelligence artificielle utilisés pour générer des prédictions. Tout d'abord, les chercheurs essayent de surmonter le défi de l'entraînement des modèles d'IA avec des ensembles de données représentatifs et diversifiés. Une sélection inadéquate des données d'entraînement peut conduire à des modèles biaisés ou peu généralisables, compromettant la validité des prédictions. Cela nécessite une collecte minutieuse et une annotation précise des données pour garantir une représentation fidèle des conditions réelles de la forêt.

\section{Systèmes de drones pour la surveillance forestière}
\subsection{Principe de la télédétection par drone}
\subsubsection{Description de la surveillance par drone}	
La technologie de surveillance par drone repose sur une approche novatrice qui exploite des aéronefs sans pilote dotés de capteurs spécialisés, offrant ainsi une méthode polyvalente et puissante pour la collecte de données à distance. Ces drones se déclinent en plusieurs types, chacun adapté à des missions spécifiques. Les drones à voilure fixe, par exemple, sont idéaux pour des missions nécessitant une couverture étendue, tandis que les drones à voilure tournante offrent une agilité accrue pour des opérations à basse altitude \cite{46}.. Il existe également des variantes hybrides qui combinent les avantages des deux types \cite{47}.
\\

Ces drones sont équipés de capteurs variés, comprenant des caméras multi spectrales, des lidars et des capteurs thermiques, permettant une acquisition de données à plusieurs niveaux spectraux. Les caméras multi spectrales sont particulièrement efficaces pour capturer des informations sur la santé des plantes et la diversité végétale, tandis que les lidars fournissent des données précises sur la topographie et la structure du terrain. Les capteurs thermiques sont utiles pour détecter les variations de température, ce qui peut être déterminant pour repérer des activités anormales, telles que des feux de forêt ou des fabrications de charbon \cite{48} , \cite{49}.

\subsubsection{Base de données et types de données à traiter}
Les drones peuvent contribuer à la génération d’une diversité de données capturées, offrant une collection d'informations pour la surveillance environnementale. Parmi ces données, on trouve des images, des nuages de points, des modèles 3D, et des informations thermiques, chacune apportant une perspective unique sur la zone d'étude. Pendant la mission, ces données peuvent être stockées à bord du drone, offrant une solution pratique pour les missions à distance ou dans des zones difficiles d'accès. Alternativement, les données peuvent être transmises en temps réel vers une station de contrôle à distance, permettant une analyse immédiate.
\\

Les images capturées par les drones, notamment les caméras multi spectrales, permettent une évaluation précise de la santé des plantes et de la biodiversité. Les nuages de points et les modèles 3D offrent une représentation détaillée du relief et de la structure du terrain, tandis que les données thermiques peuvent être significatives pour détecter les variations de température, telles que celles associées aux feux de forêt. Cette diversité de données offre une compréhension approfondie de la zone étudiée, permettant une prise de décision éclairée en matière de gestion des ressources forestières, de conservation de la biodiversité, et de surveillance des écosystèmes forestiers. Ainsi, les drones se positionnent comme des outils indispensables pour la collecte de données environnementales riches et détaillées, contribuant significativement aux efforts de préservation et de gestion durable des écosystèmes forestiers.

\subsubsection{Méthodes et étapes de traitement}
Le traitement des données issues de la télédétection par drone est un processus méthodique qui comprend plusieurs étapes. La première étape, la préparation de la mission, est fondamentale. Elle implique la planification minutieuse de la trajectoire de vol, en prenant en compte les caractéristiques spécifiques de la zone à étudier, et la sélection des capteurs appropriés pour répondre aux objectifs de la mission. Cette phase préparatoire assure le bon déroulement de la collecte des données et l'obtention d'informations pertinentes. L’étape suivante est la collecte des données pendant le vol. Les drones, équipés de capteurs variés acquièrent des données à plusieurs niveaux spectraux, offrant une vision complète de la zone d'étude. Pendant cette phase, les capteurs captent des informations spécifiques selon les besoins de la mission. 
\\

Ensuite, l'étape de traitement des données intervient après la collecte, et elle revêt une importance capitale. Elle comprend la correction des images pour garantir leur précision, la fusion des données provenant de différents capteurs pour créer une vue intégrée de la zone, et la création de modèles 3D pour une représentation détaillée de la topographie. Ces processus contribuent à la création d'un ensemble de données homogène et exploitable pour les analyses ultérieures. Enfin, l'analyse des résultats permet aux chercheurs de tirer des conclusions significatives sur la santé des écosystèmes, les changements dans le paysage, et d'autres paramètres spécifiques à la mission. Ainsi, le processus complet de la télédétection par drone, depuis la préparation de la mission jusqu'à l'analyse des résultats, constitue un outil puissant pour la compréhension et la préservation des environnements forestiers

\subsection{Application des drones pour une surveillance aérienne}
\subsubsection{Surveillance d’évolution de paysage forestier}
La surveillance de la déforestation par l'utilisation de drones équipés de caméras multi spectrales représente une avancée remarquable dans la capacité à documenter et à comprendre les changements dans les vastes étendues forestières. Ces caméras sont capables de capturer des images dans différents spectres lumineux, allant au-delà de ce que l'œil humain peut percevoir. Ce niveau de détail permet une analyse approfondie du couvert forestier, rendant possible la détection précoce des altérations. Les drones peuvent survoler des zones spécifiques à des intervalles réguliers, fournissant des séries temporelles d'images. Ces séquences d'images permettent aux chercheurs et aux responsables de comparer visuellement les changements dans le couvert forestier au fil du temps. Les caméras multi spectrales sont particulièrement précieuses car elles peuvent capter des informations sur la santé des plantes, la diversité des espèces et les signes précurseurs de la déforestation.
\\

En détectant les zones où le couvert forestier a été altéré ou éliminé, les drones peuvent identifier les zones touchées par la déforestation. Cela inclut la détection des fronts de déforestation, les zones de coupe rase, et d'autres signes visibles de l'activité humaine ayant un impact sur les forêts. En outre, les drones peuvent être utilisés pour quantifier l'ampleur de la déforestation en fournissant des données précises sur la superficie des zones touchées. Cette technologie offre un moyen efficace et précis de surveiller l'évolution des zones forestières, aidant ainsi les gestionnaires forestiers, les organismes de conservation et les autorités à prendre des décisions éclairées en matière de gestion des ressources naturelles et de préservation des écosystèmes forestiers. 

\subsubsection{Surveillance de feux de forêt}
Les drones sont équipés de capteurs thermiques et d'instruments de détection sophistiqués, ce qui en fait des outils particulièrement adaptés pour détecter rapidement et avec précision les foyers d'incendie dans des zones forestières étendues. Lorsqu'un incendie se déclare, les drones peuvent être rapidement déployés pour survoler la zone touchée et fournir des informations aux équipes d'urgence. Les capteurs thermiques des drones permettent de détecter les sources de chaleur associées aux incendies, même à travers la fumée épaisse, offrant ainsi une vision claire de la situation. Ces données sont transmises aux équipes au sol, facilitant une intervention rapide et coordonnée.
Les drones peuvent également être utilisés pour surveiller la progression du feu, en identifiant les points chauds et les zones où le risque d'expansion est plus élevé. Cette permet aux services d'incendie de prendre des décisions sur le déploiement des ressources et la mise en œuvre de stratégies pour contenir le feu. En identifiant les zones prioritaires nécessitant une intervention immédiate, les drones contribuent à maximiser l'efficacité des efforts de lutte contre les incendies. 

\subsubsection{Surveillance des animaux}
Cette méthode offre une approche non invasive, permettant d'observer les populations animales dans leur environnement naturel sans perturber leur comportement. Les drones, équipés de caméras peuvent fournir des données détaillées sur les mouvements, les interactions sociales, et d'autres aspects du comportement animal. Les images capturées par les drones permettent de documenter les schémas migratoires, les sites de nidification, et les zones concluantes pour la survie des espèces. Cette information est précieuse pour les chercheurs et les écologistes travaillant sur la protection des habitats et la préservation des corridors migratoires essentiels. Ceci dit que les drones peuvent également jouer un rôle dans la surveillance et la protection des espèces animales menacées. En identifiant les habitats critiques, les zones de reproduction, et les sites de nidification, les drones contribuent à élaborer des stratégies de conservation ciblées. En fournissant des données sur la présence d'animaux dans des zones spécifiques, ils facilitent également les efforts de lutte contre le braconnage et d'autres menaces directes.

\subsection{Avantages et limites de la surveillance par drone}
\subsubsection{Les avantages de télédétection par drone}
La télédétection par drone présente une série d'avantages significatifs, couvrant des aspects techniques, opérationnels et environnementaux. Techniquement, les drones peuvent être équipés d'une variété de capteurs, tels que des caméras multi spectrales, des lidars et des capteurs thermiques, offrant ainsi une flexibilité considérable dans la collecte de données. Cette capacité polyvalente permet une acquisition de données à plusieurs niveaux spectraux, fournissant des informations riches et détaillées sur la zone ciblée. De plus, les drones sont particulièrement adaptés pour des missions ciblées et ponctuelles, permettant la prise en compte simultanée de plusieurs paramètres grâce à l'utilisation de capteurs multiples selon les besoins spécifiques de l'étude. D'un point de vue opérationnel, la mobilité des drones est un atout majeur. Leur capacité à survoler des zones forestières denses, des terrains montagneux ou des habitats aquatiques offre une perspective aérienne qui complète efficacement les études traditionnelles menées sur le terrain. La rapidité d'intervention constitue également un avantage opérationnel. Les drones peuvent répondre en temps réel, facilitant ainsi des prises de décisions rapides et une intervention immédiate dans des situations critiques \cite{50}.
En termes environnementaux, l'utilisation de drones dans la télédétection réduit les risques liés à la présence humaine sur le terrain. Cette méthode non invasive minimise les perturbations dans les écosystèmes, offrant ainsi une approche respectueuse de l'environnement. De plus, la surveillance par drone contribue à l'efficacité des opérations de préservation en fournissant des données précises et en temps réel, permettant ainsi une gestion proactive des écosystèmes. Ces avantages combinés font de la télédétection par drone un outil puissant et adaptable pour la surveillance de la forêt et d'autres environnements naturels.

\subsubsection{Les limites de l’utilisation d’un drone}
Sur le plan technique, les drones, bien qu'ils offrent la possibilité d'être équipés de divers capteurs, se heurtent à une limitation notable de leur capacité de charge utile. Cette restriction peut influencer le nombre de capteurs ou d'instruments qu'ils peuvent transporter simultanément. Etant donné que le poids des composants, notamment la batterie et le moteur est déterminant. De plus, la portée de communication entre le drone et la station de contrôle peut être restreinte, constituant un défi, en particulier lors de l'exploration de zones éloignées. Cela s'applique tant à la transmission des commandes qu'à la transmission des données. En outre, l'intégration des données recueillies par les drones avec d'autres systèmes ou technologies peut parfois s'avérer complexe, surtout en termes de standardisation des formats de données. Il est également intéressant de noter que la transmission et le stockage des données collectées par les drones requièrent des protocoles de sécurité robustes afin de prévenir tout accès non autorisé ou violation de la vie privée. Un autre facteur essentiel à considérer est l'autonomie de la batterie, étant donné que les drones ont une durée de vol limitée en raison de la capacité de leurs batteries. Cette limitation restreint la période pendant laquelle un drone peut rester en vol, ce qui peut présenter des contraintes pour des missions de longue durée.
\\

Divers facteurs externes ont des effets impactant dans l'utilisation des drones, avec en tête les conditions météorologiques. Les drones sont particulièrement sensibles aux conditions météorologiques, telles que des vents forts, des pluies abondantes ou d'autres éléments défavorables, susceptibles d'influer sur leur stabilité et leurs performances, restreignant ainsi leur utilisation dans des situations spécifiques. De plus, le maniement d'un drone requiert généralement des compétences spécifiques pour assurer une manipulation précise de l'aéronef. Une maîtrise insuffisante pourrait compromettre la fiabilité des données collectées, à moins que le drone ne soit équipé d'un système sophistiqué d'auto-ajustement. Cependant, l'acquisition et l'entretien de drones équipés de capteurs avancés peuvent s'avérer coûteux, englobant des dépenses liées aux équipements, à la formation du personnel et à la maintenance, constituant ainsi une limitation financière. Enfin, les règlements légaux imposent des contraintes strictes sur l'utilisation des drones, en particulier dans l'espace aérien partagé. Ces restrictions légales peuvent restreindre les endroits où les drones peuvent être déployés et les altitudes auxquelles ils peuvent opérer.

\subsubsection{Défis logistiques et technologiques associés à l'utilisation des drones}
De même que lors de l’étude de la télédétection image satellite, la combinaison synergique des drones avec d'autres technologies constitue une approche innovante amplifie les capacités de surveillance et les résultats obtenus, par exemple l’intelligence artificielle et ses dérivés. De plus, la technologie des capteurs embarqués sur les drones peut être complémentaire à d'autres sources de données, l’internet des objets. En intégrant les données provenant de différentes plates-formes, on obtient une couverture plus complète et une résolution spatiale améliorée, permettant ainsi une analyse plus approfondie et précise des zones surveillées. Cette combinaison de sources de données diverses contribue à créer une image plus holistique de l'environnement étudié.
\\

Une autre synergie importante réside dans l'utilisation de la connectivité en temps réel et des systèmes de communication avancés. En combinant les drones avec des technologies de communication haut débit, les données peuvent être transmises instantanément vers des centres de contrôle distants. Cela permet une réactivité accrue, une prise de décision rapide et une coordination efficace des opérations, ce qui est essentiel dans des situations critiques, comme la détection de feux de forêt ou la réponse à des activités illégales de déforestation.

\section{Méthode manuelle et les méthodes actuellement utilisées par les gestionnaires forestiers à Madagascar}
\subsection{Méthode traditionnelle de patrouille et participative}
\subsubsection{Surveillance par des gardes forestiers}
Les stratégies actuellement mises en place à Madagascar pour assurer la préservation de cet environnement naturel sont diversifiées et rigoureuses, incluant une surveillance active à l'entrée principale du parc ainsi qu'un système de patrouille bien structuré. À l'entrée principale, des employés de la MNP veillent de manière vigilante sur toutes les personnes entrant dans le parc, qui sont majoritairement des touristes. Pour renforcer cette surveillance, chaque visiteur externe doit impérativement être accompagné d'un guide reconnu par la MNP, ce qui garantit que les visiteurs respectent les règles et prennent conscience de l'importance de la conservation. 
\\
Par ailleurs, des agents patrouilleurs de la MNP sont chargés de mener des patrouilles régulières tout au long de la journée, couvrant divers secteurs du parc pour s'assurer qu'aucune activité illégale ne se déroule. En plus de ces patrouilles programmées, ces agents sont également prêts à effectuer des patrouilles inopinées en réponse à toute activité suspecte signalée dans le parc. Cette double approche de surveillance vise à renforcer la sécurité du parc et à prévenir une variété d'activités potentiellement préjudiciables, telles que le braconnage, la déforestation illégale ou d'autres comportements nuisibles à la biodiversité de la région.
 
\subsubsection{Surveillance en collaboration avec des associations de communauté locale} 
Parmi les mesures mises en place pour assurer la surveillance des aires protégées de la MNP, figure la création d'un cadre local de collaboration clair et formel, grâce à une structure inclusive représentant les membres des communautés locales. Cette structure, connue sous le nom de Comité Local du Parc (CLP), vise à établir une cogestion des aires protégées avec les communautés locales. Ces dernières sont encouragées à travailler en étroite collaboration avec le gestionnaire du parc et à mener des activités de patrouille et de surveillance à l'intérieur des aires protégées afin de signaler toute activité illicite. De plus, le cadre juridique existant, tel que le « Dina » ou pacte local, facilite la gestion des conflits et des infractions, étant reconnu par le gouvernement comme un outil de gestion forestière ayant une force juridique.
\\

Dans le cadre du plan stratégique de la MNP, l'un des objectifs clés est la cogestion des aires protégées avec les communautés locales. Par conséquent, un soutien et un encadrement ont été apportés aux CLP à travers des formations axées sur la sensibilisation et le plaidoyer. Cette initiative a permis de former des CLP opérationnels, renforçant ainsi la collaboration entre la MNP et les communautés locales dans la surveillance et la préservation des aires protégées \cite{51}.

\subsubsection{	L’approche par collecte participative par la MBG}
Le Missouri Botanical Garden est actif à Madagascar depuis près de 47 ans et a une présence continue dans le pays depuis 1983. Depuis 2004, engagé dans la gestion environnementale, il opère à Antsiranana avec un engagement principal envers la préservation de la nature. Les actions de l'organisme se concentrent sur la recherche liée aux plantes de Madagascar et à leurs applications, englobant des activités telles que la recherche, la préservation et le développement \cite{52}. En plus de la méthode traditionnelle de patrouille pour la surveillance de la forêt de la Montagne des Français, l'organisme fait également usage du système Global Forest Watch, fondé sur la combinaison de la télédétection et l’approche de la collecte participative ou du participatory sensing.
\\

Global Forest Watch (GFW) est une plateforme mondiale de surveillance des forêts créée par le World Resources Institute (WRI). Son objectif principal est de fournir en temps réel des informations sur l'état des forêts à l'échelle mondiale en utilisant des données satellitaires, des analyses de télédétection, des technologies avancées et la participation du public. La plateforme permet de suivre divers aspects des forêts, tels que la couverture forestière, la déforestation, les incendies, les concessions forestières, et d'autres indicateurs importants \cite{53}. Elle offre une visualisation interactive des données, permettant aux utilisateurs de surveiller les changements et d'accéder à des informations détaillées sur des zones spécifiques, tout en recevant des alertes en temps réel sur des activités potentiellement préjudiciables aux forêts \cite{54}.

\subsection{Limites des méthodes actuelles appliquées}
\subsubsection{Problèmes liés à la méthode de patrouille}
La mise en place d'unités de patrouille a permis de réduire les activités illégales dans les zones protégées, mais leur efficacité reste perfectible. Face à un contexte en constante évolution, les braconniers développent des stratégies pour contourner les patrouilles. De plus, la méthode traditionnelle de patrouille dans des réserves naturelles rencontre des défis significatifs en raison de l'immensité de la zone à surveiller, couvrant des milliers d'hectares de forêt dense et de terrain accidenté. Il est souvent impossible de couvrir de manière exhaustive l'ensemble de l'environnement, ce qui rend les surveillances aléatoires et irrégulières. 
\\

Les patrouilleurs sont confrontés à une grave insuffisance d'équipements adaptés : ils ne disposent pas de véhicules nécessaires pour naviguer dans des zones inaccessibles autrement, ni de matériel de communication pour coordonner leurs efforts et signaler rapidement les incidents. De nombreuses parties de la forêt sont inaccessibles par des chemins directs, obligeant les patrouilleurs à parcourir des heures à pied à travers des sentiers étroits et escarpés, augmentant ainsi la fatigue et réduisant leur efficacité. Cette inaccessibilité complique encore la tâche de surveillance, car certaines zones critiques peuvent ne jamais être atteintes. De plus, ils manquent de kits de sécurité, ce qui les expose à des risques personnels élevés. Le défi majeur réside donc dans la nécessité de moderniser le système de surveillance pour assurer une protection efficace et continue \cite{55}.

\subsubsection{Limite de l’outils de GFW }
Le Global Forest Watch présente des limites dans la surveillance de la déforestation en raison de plusieurs facteurs. Premièrement, il est difficile de distinguer les véritables forêts du couvert forestier agricole, tel que les plantations de palmiers à huile, ce qui peut entraîner une surestimation de la superficie forestière par les systèmes de surveillance par satellite. De plus, l'absence d'ensembles de données mondiaux sur l'utilisation des terres rend difficile la distinction entre la perte permanente de couvert arboré, liée à la déforestation, et les pertes temporaires dues à d'autres facteurs tels que les incendies de forêt ou les rotations de récolte de bois.
\\

En outre, le couvert arboré est une mesure unidimensionnelle qui ne capture pas toutes les caractéristiques d'une forêt, ce qui complique la détection de la dégradation forestière. Les forêts présentant des différences significatives en termes de forme et de fonction, telles que les forêts primaires intactes et les plantations gérées pour la production de bois, sont presque impossibles à distinguer sur les images satellites basées uniquement sur le couvert forestier. De plus, le gain de couvert arboré est plus difficile à mesurer que la perte, car il s'agit d'un processus progressif difficile à discerner d'une image satellite à l'autre. Enfin, les variations méthodologiques et temporelles dans les données sur le couvert arboré, la perte et le gain rendent difficile la comparaison et le calcul précis de la perte nette de couverture arborée \cite{56}.

\section{Conclusion}
\subsection{Synthèse sur les enjeux de la surveillance de la déforestation}
En conclusion, la surveillance de la déforestation revêt une importance déterminante en raison des enjeux liés à l'abattage d'arbres. On a constaté que la déforestation a des conséquences importantes sur la nature. Elle perturbe le climat en relâchant des gaz qui contribuent au réchauffement de la planète, accentuant ainsi les problèmes liés aux changements climatiques. De plus, elle rend le sol plus vulnérable, réduisant sa fertilité et l'espace disponible pour cultiver des plantes. Les forêts, qui sont importantes pour la conservation de l'eau, voient également leur capacité à influencer les cycles de pluie diminuer, ce qui peut affecter l'accès à l'eau potable. La déforestation menace aussi la diversité des plantes et des animaux, mettant en danger l'équilibre de la vie dans les forêts. 
Sur le plan économique, elle peut nuire au tourisme, réduire les revenus des habitants locaux et avoir d'autres conséquences financières. Cependant, une gestion responsable des forêts peut offrir des avantages économiques en préservant la diversité naturelle et en renforçant la réputation internationale d'un pays. Dans le nord de Madagascar, la coupe due à la culture excessive d'une plante appelée Catha edulis et l'exploitation des arbres pour faire du charbon de bois et des meubles sont des causes importantes de déforestation.
\\

Ainsi, la surveillance forestière aide dans la préservation de l'environnement en empêchant une érosion accélérée et en assurant la santé des sols, une base fondamentale pour notre existence. Elle permet d'évaluer l'étendue des activités de coupe, d'anticiper les impacts sur les cycles hydrologiques et de mettre en œuvre des mesures de conservation spécifiques. Cette surveillance continue se révèle être un outil indispensable pour assurer la préservation des espèces végétales et animales. En agissant comme un rempart essentiel, la surveillance forestière permet de contrôler ces activités et de prévenir la déforestation persistante dans cette région.
\\

\subsection{Synthèse sur les technologies associées à la surveillance de la déforestation}
D’une part, nous avons trouvé la télédétection par imagerie satellite. Elle constitue une méthode efficace pour la surveillance forestière, reposant sur des composants clés tels que la source d'énergie, le rayonnement atmosphérique, l'objet d'étude, le capteur enregistreur, le système de traitement des données, et l'analyse interprétative. Divers capteurs, tels que les optiques, radars, hyper spectraux, thermiques et lidar, sont utilisés pour collecter des données traitées à des fins de surveillance environnementale. En particulier, la télédétection offre des applications spécifiques dans la surveillance des changements forestiers, la détection de déforestation illégale, le suivi de la croissance des plantations, et la cartographie des feux de forêt. Les avantages de cette méthode incluent sa portée globale, sa continuité temporelle, sa sécurité des données, et sa capacité à minimiser les perturbations dans des zones difficiles d'accès. Cependant, les limites se trouvent dans la résolution, la dépendance aux conditions atmosphériques, la difficulté à l’accès aux données et la nécessité d’une haute capacité de traitement.
\\

D’autre part, les systèmes de drones pour la surveillance forestière représentent une approche polyvalente basée sur l’utilisation d’un aéronef sans pilote humain à bord. Elle permet la collecte de données à distance et fournissant des informations détaillées sur les environnements forestiers. Grâce à une variété de capteurs embarqués, tels que les caméras multi spectrales, les lidars et les capteurs thermiques, ces drones acquièrent des données à plusieurs niveaux spectraux. La base de données résultante, comprenant des images, des nuages de points, des modèles 3D et des données thermiques, constitue une ressource précieuse pour la surveillance environnementale. Cette base permet l'évaluation de la biodiversité, la prédiction de la déforestation, la surveillance des incendies de forêt et l'observation des populations animales. On constate que les applications des drones offrent des avantages significatifs pour la gestion des ressources naturelles, la conservation de la biodiversité et la protection des écosystèmes. Du fait de volet à des altitudes basses, les drones peuvent acquérir une résolution plus élevée et une capture détaillée d’une zone spécifique. Son utilisation est plus adaptée à la surveillance de petites zones pour une cartographie plus détaillée. Son avantage réside sur la possibilité d’une capture d'image à la demande, offrant une flexibilité temporelle pour des missions spécifiques. Mais par contre, ses coûts peuvent être plus élevés du point de vue matériel mais aussi sur la nécessité d’une personne qualifiée pour sa manipulation. De plus, ils sont aussi affectés par les conditions météorologiques bien que moins par rapport à la télédétection par satellite. Les drones sont spécifiquement déployés pour une surveillance plus ciblée et prédéfinit, mais présente une difficulté sur la surveillance précoce. Enfin, ils sont plus Idéals pour des applications telles que la cartographie de précision, la surveillance agricole où on a besoin de données plus détaillées.
\\
Les initiatives de conservation de la Montagne d'Ambre comprennent la surveillance de l'entrée principale et un système de patrouille impliquant des agents et gardes forestiers, dont certains résident dans le parc. L'objectif est de prévenir le braconnage et la déforestation. Par contre la surveillance reste un défi en raison de la vaste étendue de la zone, exposant la région à des risques potentiels. Parallèlement, le MBG, actif dans la gestion environnementale à Antsiranana depuis 2004, se concentre sur la préservation de la nature. En complément de la patrouille traditionnelle, le MBG utilise GFW, une plateforme mondiale basée sur la collecte participative, intégrant des données satellitaires et la participation du public pour fournir des informations en temps réel sur les forêts à l'échelle mondiale. Néanmoins, les méthodes actuelles de surveillance, notamment la patrouille traditionnelle, sont entravées par la vaste étendue à couvrir, entraînant une surveillance inégale. De plus, la surveillance participative pose des défis locaux car la population n'est pas familière avec cette méthode, compliquant la collecte de données de qualité par des participants non experts et exposant la Montagne d'Ambre à des risques accrus.

\subsection{Motivation sur la surveillance par capteurs aux sols}
Dans le cadre de notre recherche approfondie, nous avons centré notre attention sur la surveillance précoce de la déforestation. Cette approche se distingue nettement de la télédétection, qui se concentre principalement sur la détection des dommages forestiers, l'analyse et la prédiction intervenant souvent après le commencement de la déforestation. Notre premier objectif est orienté vers la prévention, consistant à identifier les signes avant qu'une déforestation ne débute. Cette démarche s'inscrit dans la perspective de la protection des arbres en détectant les coupes, soulignant ainsi l'importance de chaque arbre individuel. D'où la nécessité impérieuse d'effectuer une surveillance en temps réel pour garantir une réactivité optimale.
\\

Un deuxième objectif clé de notre étude réside dans la quête de précision dans la surveillance. Nous aspirons à développer des méthodes et des technologies qui permettent une identification minutieuse des activités de déforestation, offrant ainsi une vision détaillée des zones potentiellement menacées. Un autre axe de notre recherche se concentre sur la réduction des coûts associés à la surveillance anticipée de la déforestation. Nous cherchons des solutions innovantes et efficaces qui permettent d'optimiser l'utilisation des ressources tout en maintenant une performance élevée. En outre, un quatrième objectif convaincant est d'assurer la pérennité et la fiabilité des systèmes de surveillance déployés dans le temps, garantissant ainsi une protection continue des écosystèmes forestiers.
\\

Pour atteindre ces objectifs ambitieux, nous avons examiné attentivement la détection par capteur au sol, visant à fournir des réponses en temps réel pour une intervention rapide et efficace. En outre, l'examen de l'alimentation par énergie renouvelable dans le contexte de ces systèmes de surveillance fait partie intégrante de notre démarche pour assurer une durabilité environnementale globale de nos solutions. Parallèlement, nous avons délibérément choisi d'adopter une approche en détectant la déforestation par le biais de la détection sonore. Cette méthode repose sur l'utilisation de l'intelligence artificielle pour le traitement de données, garantissant une précision accrue dans l'identification des activités potentiellement préjudiciables.