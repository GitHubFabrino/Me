\chapter*{Liste des Publications et Communications}
\phantomsection
\addcontentsline{toc}{chapter}{Liste des Publications et Communications}

Le travail présenté dans cette thèse a donné lieu à un certain nombre de publications et de communications :

\begin{enumerate}
\item	Aicha Yvanna Rasoarimanana, Manankaja Rongatry Mahazomila, Rovamanjaka Onjamalala Lucas Rollandros Ravoniharinaivo, \textbf{Hery Tina Ramanan’haja}, Hasina Andrianina Rakotonirina Fitiavana Andriamiharinjaka, Youssef Kebbati, Odette Fokapu, Smart Mini Solar Dryer for Laboratory Pilot, International Journal Of Engineering Research \& Technology (Ijert), Volume 13, Issue 05 (May 2024).\\
	
	\item 	Raonirivo N. Rakotoarijaona, Mamisoa Randriamparany, \textbf{Hery Tina Ramanan’haja}, Rakotobe Tefy Raoelivololona, Cartographie par images satellites des foyers de mangroves à Madagascar dans Google Earth Engine, Colloque international à l’Université d’Antsiranana sur les enjeux sociétaux et technologiques à l’épreuve du développement numérique dans l'océan Indien et en Afrique. Antsiranana, Madagascar, Mars 2024.\\
	
\item	\textbf{Hery Tina Ramanan’haja}, Youssef Kebbati, Damien Audoux, Odette Fokapu, Rakotobe Tefy Raoelivololona, Jean Marie Razafimahenina, 2023, Evaluating Energy and Sizing of Electronic System For Environmental Monitoring In Madagascar Forest, International Journal Of Engineering Research \& Technology (IJERT), Volume 12, Issue 11 (November 2023).\\
	
\item	\textbf{Hery Tina Ramanan’haja}, Maheritiana Jonathan Jérémie Randriarison, Rakotobe Tefy Raoelivololona, Odette Fokapu, Youssef Kebbati, Jean Marie Razafimahenina, 2023, Environmental monitoring by sound source detection using machine learning, International Journal Of Engineering Research \& Technology (IJERT), Volume 12, Issue 10 (October 2023).\\
	
\item	\textbf{Hery Tina Ramanan’haja}, Antonio Prosper Zara, Rakotobe Tefy Raoelivololona, Jean Marie Razafimahenina, Analyse de source sonore pour une prévention des infractions forestières par intelligence artificielle. Cas d’étude : détection de découpe d’arbre, 6ème édition des Doctoriales sur le développement durable du monde rural, Antsiranana, Décembre 2020.\\
	
\item	Mamisoa Randriamparany, Justin Ratsaramody, Michel Aime Randriazanamparany, \textbf{Hery Tina Ramanan’haja}, Cartographie et évaluation rapides des dégâts d’une inondation avec des données gratuites et logiciels libres : cas de la zone inondable du Sambirano, Madagascar, Afrique Science, ISSN 1813-548X, Vol 15, Issue 2, 2019, pp 24-31.
	
\end{enumerate}

